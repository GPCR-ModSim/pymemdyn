%===============================================================================
%  Preamble
%===============================================================================
\documentclass[10pt, oneside, pdftex]{article}
\usepackage[T1]{fontenc}
\usepackage[bitstream-charter]{mathdesign}
\usepackage[latin1]{inputenc}			% Input encoding
\usepackage{amsmath, amstext, amsfonts}		% Math
\usepackage{xcolor}
\definecolor{bl}{rgb}{0.0,0.2,0.6} 
\usepackage{sectsty}
\usepackage{url}
\usepackage{fancyvrb}
\usepackage{listings}
\usepackage[pdftex]{hyperref}%%Note: In TeXShop hyperref needs to be at the end of package calls.
\usepackage{setspace}
\singlespacing
\usepackage[left=2.54cm,bottom=2.00cm,top=2.00cm,right=2.54cm]{geometry}
\usepackage[compact]{titlesec} 
\allsectionsfont{\color{bl}\scshape\selectfont}

%===============================================================================
%  Definitions
%===============================================================================
% Define a new command that prints the title only
\makeatletter						% Begin definition
\def\printtitle{%					% Define command: \printtitle
{\color{bl} \centering \huge  \textbf{\@title}\par}}	% Typesetting
\makeatother						% End definition

\title{py-MEMdyn \\ 
\large \vspace*{-10pt}Python Membrane Dynamics\vspace*{10pt}}

% Define a new command that prints the author(s) only
\makeatletter						% Begin definition
\def\printauthor{%					% Define command: \printauthor
{\centering \small \@author}}				% Typesetting
\makeatother						% End definition

\author{%
\textit{Python - Membrane Dynamics} \\
A Python library for Molecular Dynamics simulations in biological membranes\\
Manual V. 1.1 \\
Xabier Bello, Mauricio Esguerra, David Rodriguez, Hugo Guti\'{e}rrez de Ter\'{a}n \\
hugo.gutierrez@icm.uu.se \\
\vspace{20pt}
}

% Custom headers and footers
\usepackage{fancyhdr}
\pagestyle{fancy}					% Enabling custom headers/footers
\usepackage{lastpage}	

% Header (empty)
	\lhead{}
	\chead{}
	\rhead{}

% Footer (you may change this to your own needs)
	\lfoot{\footnotesize \texttt{gpcr-modsim.org} - }
	\cfoot{}
	\rfoot{\footnotesize page \thepage\ }%of \pageref{LastPage}}	% "Page 1 of 2"
	\renewcommand{\headrulewidth}{0.0pt}
	\renewcommand{\footrulewidth}{0.4pt}

% Change abstract environment
\usepackage[runin]{abstract}		% runin option for a run-in title
\setlength\absleftindent{30pt}		% left margin
\setlength\absrightindent{30pt}		% right margin
\abslabeldelim{\quad}			% 
\setlength{\abstitleskip}{-10pt}
\renewcommand{\abstractname}{}
\renewcommand{\abstracttextfont}{\color{bl} \small \slshape}	% slanted text


%===============================================================================
% Start document
%===============================================================================
\begin{document}
% Top of the page: Author, Title and Abstact
\printtitle 
\printauthor

\begin{abstract}
\noindent \textbf{py-MEMdyn}  is a python library  written to automate
the process of setting up the simulation, via Molecular Dynamics (MD),
of G-protein  Coupled Receptors (GPCR's) embedded in  a cell membrane.
The protocol  can be adapted  to insert other  transmembrane proteins,
not only GPCR's.  The protocol has been adapted from that described in
Guti\'{e}rrez de  Ter\'{a}n et al.  (2011) \cite{rodriguez2011}, and is  implemented in
the web-based service for  modeling and simulation of GPCR's available
at \url{http://gpcr-modsim.org}.

This  fully automated  pipeline allows  any researcher,  without prior
experience in computational chemistry, to perform an otherwise tedious
and   complex  process   of   membrane  insertion   and  thorough   MD
equilibration, as outlined  in Fig. 1. In the  simplest scenario, only
the  receptor structure  is  considered.  This  will be  automatically
surrounded by a  pre-equilibrated POPC membrane model in  a way that the
TM bundle  is parallel  to the membrane  vertical axis. The  system is
then soaked with bulk water and inserted into a hexagonal-prism shaped
box,  which  is energy-minimized  and  carefully  equilibrated in  the
framework of periodic boundary conditions (PBC). It follows a thorough
MD equilibration protocol that lasts  2.5 ns. But the simulation of an
isolated receptor  can only account for  one part of  the problem, and
the influence  of different non-protein elements  in receptor dynamics
such  as the  orthosteric ligand,  an allosteric modulator,  or even
specific cholesterol, lipid, water or  ion molecules is key for a more
comprehensive characterization of GPCR's. To allow a broader use of MD
simulations  to  researchers  in  the  field,  \textbf{py-MEMdyn}  can
explicitly handle these elements.  Each molecule should be uploaded as
docked to the original PDB file  of the receptor, so they are properly
integrated  into  the  membrane  insertion protocol  described  above,
together with  the force-field associated  files (which can  be either
generated  with external  software like  Macromodel, either  by manual
parameterization).   In addition, it  is also  possible to  perform MD
simulations of  receptor dimers,  provided that a  proper dimerization
model  exists,  i.e., coming  from  X-ray  crystallography  or from  a
protein-protein docking  protocol.  The  ease of use,  flexibility and
public  availability  of the  \textbf{py-MEMdyn}  library  makes it  a
unique tool for researchers in  the GPCR field interested in exploring
dynamic processes of these receptors.
\end{abstract}


\section*{ - Installation}
\textbf{py-MEMdyn} is a python library thaht can be used in  any unix
platform provided that the following dependencies are installed:
\begin{itemize}\itemsep0em
\item {A unix-line text terminal.}
\item {ssh for secure-shell connection.}
\item {git (repository access).}
\item {Python (required >= 2.7).}
\item {Gromacs 4.0.5, 4.6.5 supported.}
\item {Queuing  system:  although  not  strictly  required,  this  is  highly
advisable since  a 5 ns MD  simulation will be  performed. However, if
only  membrane insertion  and energy  minimization is  requested, this
requirement can  be avoided. Currently, the  queuing systems supported
include Slurm, PBS.}
\end{itemize}

In  order to  install the  last version,  the user  must  have granted
access  to  the  repository,  located  at the  cluster  ``Cuelebre''  at
USC. The following steps are  needed though. It is assumed a unix-like
terminal and  that the ssh  protocol for communications is  enabled in
the  user  computer.  

\begin{enumerate}
\item{Have  a user  created  in  a machine  called
``grupomaside'' at USC, which is used  as a tunel to access our cluster,
Cuelebre.  You also  need a  user un  Cuelebre (from  now on,  we will
assume that both are the same username).}

\item{Create a  tunnel through  grupomaside: 
\begin{Verbatim}
ssh  -f -N  -l  username -L 7777:cuelebre.inv.usc.es:22 grupomaside.usc.es -p922 
\end{Verbatim}
(Adding  the -f -N  flags to  that line  will force  the tunnel  to go
background)}

\item{(This  step  only needed  the  first  time  that we  download  the
py-MEMdyn program) Create a local  copy of the repository: Situate the
cursor  in  the local  directory  where you  want  the  program to  be
installed  (we  will  assume  /home/localuser)  and  type:
\begin{Verbatim}
git  clone ssh://username@localhost:7777/home/slurm/pymoldyn/share  pymoldyn/  
\end{Verbatim}
If your  local  user  is  different  from the  trasgu  user,  adding  the
``username@'' is mandatory,  as Git is working through  the tunnel using
the local user name}

\item{For every update, locate the cursor IN the local pymoldyn directory
and type: git pull  Or of you are on a remote  computer you might need
to       put       the        full       options:
\begin{Verbatim}
git pull ssh://localhost:7777/home/slurm/pymoldyn/share
\end{Verbatim}
}

\item{Make sure  that your  gromacs installation  is understood  by the
program. To  specify the path of  Gromacs, edit the file  (with a text
editor, i.e.. the  Unix program ``vi'') settings.py, and  make sure that
only   one  line   is  uncommented,   looking  like:   GROMACS\_PATH  =
/opt/gromacs405/bin Provided that in your case gromacs is installed in
/opt. The  program will  prepend this line  to the binaries  names, so
calling ``/opt/gromacs405/bin/grompp'' should point to that binary.}

\item{Similarly,  in that file you  specify which queuing  system you are
using.  We will  assume  that you  will  use ``slurm''  at the  cuelebre
cluster. In this case, you will login there through the tunel. This is
a 2 stepwise process, the first one is only done ONCE in your session,
since it  opens a tunel  that will only  be closed when  you disconect
your workstation from the computer, which is STEP 2 in this guide. The
second is justy an ssh through this tunel to Cuelebre: ssh -l username
-p 7777 localhost  Indeed I have an alias in  my .bash\_profile to make
this connection from my workstation: alias tunel1='ssh -f -N -l hteran
-L   7777:cuelebre.inv.usc.es:22   grupomaside.usc.es  -p9222'   alias
tunel2='ssh -l username -p 7777 localhost'

And I simply  run the commands ``tunel1'' (it will  prompt for my passwd
in grupomaside)  and thereafter  tunel2 (again it  will prompt  for my
passwd in Cuelebre). It is a good idea to have both passwd the same to
avoid confusions}
\end{enumerate}

\begin{itemize}
\item{COMPULSORY! Option  -p: In  the simplest case,  Pymoldyn only
  needs  a pdb  file with  the receptor.  This should  be  readable by
  Gromacs (i.e., accomplish the  PDB standards, see the GROMACS manual
  for details)  and accessed  from the working  directory. And  as the
  error says, this is the bare minimum to perform an MD simulation! we
  will assume that the file is called gpcr.pdb Thus run.sh -p gpcr.pdb
  should work!  Accesory options: The common user should not take care
  of the -b, -r or --debug options: The working directory (OWN\_DIR) is
  set  by  default to  that  where the  program  is  invoked, and  the
  repository  (REPO\_DIR)  is specified  in  the  settings.py file  (by
  default,  templates/  subdirectory   in  the  pymoldyn  instalation).}
  
\item{Considering  non-protein  elements:  ligand(s),  structural  waters,
  structural lipids, cholesterol molecules, explicit ions.}
\begin{itemize}
\item{Option  -l (specifying an  orthosteric ligand). Lets assume  that we
have docked a  ligand in the orthosteric binding  site. We can include
this  in the simulation  as long  as we  have generated  the requested
library and parameter files in gromacs. Thus, 3 files are needed, that
should share a root name (i.e., lig):}
\begin{itemize}
\item{lig.pdb:  a standard  pdb file  where  the atom  names are  explicitly
considered in the  itp and ff files (see bellow)}
\item{lig.itp: we refer to
as  the library file,  and collects  the atom  charges and  the bonded
parameters (i.e.,  bonds, angles,  dihedrals and torsions)  as derived
with  the  OPLS  forcefield  in  the  Gromacs  standard  nomenclature.}
\item{lig.ff: we will refer to as the ``force-field file'', which collects the
OPLS2005 atom types and  non-bonded parameters in the Gromacs standard
nomenclature. For  the users used  to Gromacs, this file  is generally
non-existing  and  the parameters  listed  here  are  merged onto  the
standard forcefield  file in gromacs  (i.e. ffoplsaa.itp). But  in our
protocol, this  is needed as a  separate file.}
run.sh  -p gpcr.pdb -l lig
\end{itemize}
\item{Option --alo  (specifying an allosteric ligand):  This should be
  treated exactly the same as  the ligand: if the allosteric modulator
  is  called allo,  we  need  to have  alo.pdb,  alo.itp and  alo.ff.}
  run.sh -p gpcr.pdb --alo alo 
\item{Option --water (specifying structural
  waters):  If structural  waters are  present (i.e.,  coming  from an
  x-ray structure)  these should  be included as  a separate  pdb file
  (i.e. hoh.pdb). The corresponding itp file (hoh.itp) is also needed,
  but in this case the ff file is avoided as the parameters are in the
  standard forcefield.}   
\item{Option --ions  (specifying structural ions)
  If we want to consider structural  ions (i.e., the sodium ion in the
  A2A high resolution structure 4EIY), we should name the needed files
  i.e. ion-local  and provide the same  information as in  the case of
  structural waters: ion-local.pdb  and ion-local.itp.}  
\item{Option --cho
  (specifying cholesterol molecules): As in the previous case, the pdb
  and  itp files  are needed  (i.e., cho.pdb  and cho.itp).}
\item{Option --queue:  if other  queue than  the  default one,  indicated in  the
  settings.py,  is needed  (in our  example, slurm  in  cuelebre) this
  should be indicated  in the option. Possibilities are  listed in the
  settings.py  file:}
\begin{itemize}
\item{slurm  as  implemented in  cuelebre}
\item{pbs  as  implemented  in   garibaldi.scripps.edu}
\item{pbs\_ib   infiniband,  as implemented  in  garibaldi.scripps.edu}
\item{svgd, as  implemented  in svgd.cesga.es}
\end{itemize}
\end{itemize}
\end{itemize}

To summarize, the following command should work in the most complex case,
run.sh -p gpcr.pdb -l lig --waters hoh --alo alo --cho cho

\section*{Running with queues}
99\% of the time you will want  to use some queue system. We deal with
queue systems tweaks as we stumble into them and it's out of our scope
to cover them all.  If you take a look at the  source code dir, you'll
found some  files called ``run\_pbs.sh'',  ``run\_svgd.sh'' and so  on. Also
there are specific  queue objects in the source  file queue.py we have
to tweak for every and each queue.  In you want to run your simulation
in a supported queue, copy the ``run\_queuename.sh'' file to your working
directory, and edit it. E.g.  the workdir to run an A2a.pdb simulation
in svgd.cesga.es looks like:  . .. A2a.pdb run\_svgd.sh And run\_svgd.sh
looks   like:  
\begin{Verbatim}
\$!/bin/bash   
module   load  python/2.7.3
module   load gromacs/4.0.7
python ~/bin/pymoldyn/run.py  -p  a2a.pdb 

Now  we  just launch this script with: 
qsub -l arch=amd,num_proc=1,s_rt=50:00:00,s_vmem=1G,h_fsize=1G -pe mpi 8
run_svgd.sh and wait  for the results. Note that  we launch 1 process,
but flag the run as mpi with reservation of 8 cores in SVGD queue.
\end{Verbatim}

\section*{ - Debugging} 
If you are to set up a new system, it is a good idea to just
run a  few steps of each  stage in the equilibration  protocol just to
test that  the pdb file  is read correctly and  the membrane-insertion
protocol  works fine  and the  system can  be minimized  and  does not
``explode''  during the  equilibration  protocol (i.e.,  detect if  atom
clashes and so on exist on your system).

To do this, use the --debug option, like:
\begin{Verbatim}
run.sh -p gpcr.pdb -l lig --waters hoh --debug 
\end{Verbatim}

If everythings works fine, you will see the list of output directories
and files just  as in a regular equilibration  protocol, but with much
smaller files (since we only use here 1000 steps of MD in each stage).
NOTE  that  sometimes, due  to  the  need  of a  smooth  equilibration
procedure (i.e.  when a new ligand  is introduced in  the binding site
without further refinement  of the complex, or with  slight clashes of
existing  water  molecules)   this  kind  of  debugging  equilibration
procedure  might crash during  the first  stages due  to hot  atoms or
LINCS failure. This is normal, and you have two options: i) trust that
the full equilibration  procedure will fix the steric  clashes in your
starting  system,  and then  directly  run  the  pymoldyn without  the
debugging option, or  ii) identify the hot atoms  (check the mdrun.log
file  in  the  last  subdirectory  that was  written  in  your  output
(generally eq/mdrun.log  and look for  ``LINCS WARNING'').  What  if you
want to check partial functions of  pymoldyn?  In order to do this you
must edit  the file run.py  and change: 
\begin{enumerate}
\item{Line 211 comment  with ``\#''
this line  [that states: run.clean()],  which is the one  that deletes
all the output files present in the working directory.}
\item{In the  last two  lines of  this file, comment  (add a ``\#'') the
  line: run.moldyn()}
\item{And uncomment (remove the ``\#'') the line: run.light\_moldyn()}
\item{In the line  140 and within that block  (ligh\_moldyn) change the
  lines stating steps = ["xxxx"] and include only those steps that you
  want to test, which should be within a list of strings.}
\end{enumerate}

For the  sake of clarity,  these have been  subdivided in  two lines:  

\begin{Verbatim}
line 1-  steps = ["Init", "Minimization", "Equilibration", "Relax", "CARelax"] 
\end{Verbatim}
Here you remove those strings that you do not want to be executed, i.e. if only
membrane insertion and minimization is wished, remove "Equilibration",
"Relax",   "CARelax"   so   the   line  states:   
\begin{Verbatim}
steps   =   ["Init",  "Minimization"]  
\end{Verbatim}

\begin{Verbatim}
line  2-   steps  =  ["CollectResults"]  
\end{Verbatim}

This  only accounts for the  preparation of the output files  for analysis, so if
you only are interested on  this stage, comment the previous line. The
last assignment  is the one that  runs.  NOTE that you  must know what
you do, otherwise  you might have crashes in the  code if needed files
to  run  intermediate  stages   are  missing!   

\section*{ - Output}  
The  performed equilibration includes the following stages:

\begin{table}[htbp]
\centering
\small\addtolength{\tabcolsep}{-2pt}
\begin{tabular}{p{1.6cm}|c|c|p{2.2cm}}
\hline
\textbf{STAGE} & CONSTRAINED ATOMS & FORCE CONSTANTS & TIME (ns)  \\ \hline
Minimization   &	 	   &                 & (Max. 500 steps)\\
Eq1            & Protein Heavy Atoms & 1000	     & 0.5\\
Eq2            & Protein Heavy Atoms & 800	     & 0.5\\
Eq3            & Protein Heavy Atoms & 600	     & 0.5\\
Eq4            & Protein Heavy Atoms & 400	     & 0.5\\
Eq5            & Protein Heavy Atoms & 200	     & 0.5\\
Eq6            & Protein Heavy Atoms & 200	     & 2.5\\
\end{tabular}
\parbox{5.8in}{\caption{\footnotesize{Simulation steps performed by default using pyMEMdyn}}}
\label{tab:equilibration}
\end{table}
% $ kJ *mol^{-1}*nm^{-2}$	ns



In this folder you will find several files related to this simulation:

\begin{Verbatim}
INPUTS:
- popc.itp              # Topology of the lipids
- ffoplsaa_mod.itp      # Modified OPLSAA-FF, to account for lipid modifications
- ffoplsaabon_mod.itp   # Modified OPLSAA-FF(bonded), to account for lipid modifications
- ffoplsaanb_mod.itp    # Modified OPLSAA-FF(non-bonded), to account for lipid modifications
- topol.tpr             # Input for the first equilibration stage
- topol.top             # Topology of the system
- protein.itp           # Topology of the protein
- index.ndx             # Index file with appropriate groups for GROMACS
- prod_example.mdp      # Example of a parameter file to configure a production phase (see TIPS)

STRUCTURES:
- hexagon.pdb   	# Initial structure of the system, with the receptor centered in the box
- confout.gro   	# Final structure of the system (see TIPS)

TRAJECTORY FILES
- traj_EQ.xtc   	# Trajectory of the whole system in .xtc format: 1 snapshot/50 ps	 
- ener_EQ.edr   	# Energy file of the trajectory

REPORTS:
In the "reports" subfolder, you will find the following files:
- tot_ener.xvg, tot_ener.log # System total energy plot and log
- temp.xvg, temp.log         # System temperature plot and log
- pressure.xvg, pressure.log # System pressure plot and log
- volume.xvg, volume.log     # System volume plot and log

LOGS:
In the "logs" subfolder, you will find the log files of mdrun:
- eq_{force_constant}.log    # log of stages with restrained heavy atoms of the receptor
- eqCA.log                   # log of the stage with restrained C-alfa atoms of the receptor
\end{Verbatim}

\section*{ - TIPS}
\begin{itemize}
\item{If you want to configure a .tpr input file for production phase, you
can use the  template 'prod.mdp' file by introducing  the number steps
(nsteps), and thus  the simulation time, you want  to run. After that,
you just have to type:
\begin{Verbatim}
grompp -f prod.mdp -c confout.gro -p topol.top -n index.ndx -o topol_prod.tpr
\end{Verbatim}
}

\item{If  you  want  to  create  a  PDB file  of  your  system  after  the
equilibration, with the receptor centered in the box, type: echo 1 0 |
trjconv  -pbc mol -center  -ur compact  -f confout.gro  -o confout.pdb
NOTE: these tips work for GROMACS version 4.0.5.}
\end{itemize}

\begin{thebibliography}{10}
\bibitem{rodriguez2011} D.  Rodr\'{i}guez,  A.  Pi\~{n}eiro,  and H. Guti\'{e}rrez-de-Ter\'{a}n. 
2011, Molecular dynamics simulations reveal insights into key structural
elements  of  adenosine  receptors. \textit{Biochemistry}, \textbf{50}, 4194-4208.
\end{thebibliography}








%\lstset{language=python, frame=single}
%\fvset{frame=single, fontfamily=courier, fontsize=\small}
%\begin{Verbatim}
%color white
%color blue, (pc; > 0.1)
%color red,  (pc; < 0.1)
%show surface
%\end{Verbatim}

\end{document}
